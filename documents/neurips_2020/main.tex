\documentclass{article}

% if you need to pass options to natbib, use, e.g.:
%     \PassOptionsToPackage{numbers, compress}{natbib}
% before loading neurips_2019

% ready for submission
% \usepackage{neurips_2019}

% to compile a preprint version, e.g., for submission to arXiv, add add the
% [preprint] option:
\usepackage[preprint]{neurips_2019}

% to compile a camera-ready version, add the [final] option, e.g.:
% \usepackage[final]{neurips_2019}

% to avoid loading the natbib package, add option nonatbib:
%     \usepackage[nonatbib]{neurips_2019}

\usepackage[utf8]{inputenc} % allow utf-8 input
\usepackage[T1]{fontenc}    % use 8-bit T1 fonts
\usepackage{hyperref}       % hyperlinks
\usepackage{booktabs}       % professional-quality tables
\usepackage{amsfonts}       % blackboard math symbols
\usepackage{amsmath}
\usepackage{float}
\usepackage{graphicx}
\usepackage{lipsum}
\usepackage{microtype}      % microtypography
\usepackage{nicefrac}       % compact symbols for 1/2, etc.
\usepackage{tikz}
\usetikzlibrary{bayesnet}
\usepackage{url}            % simple URL typesetting
\usepackage{wrapfig}
\usepackage{xcolor}

% might be necessary to suppress References created by \bibliography{}
\usepackage{biblatex}
\addbibresource{biblio.bib}

\title{IBL RNN Project}

% The \author macro works with any number of authors. There are two commands
% used to separate the names and addresses of multiple authors: \And and \AND.
%
% Using \And between authors leaves it to LaTeX to determine where to break the
% lines. Using \AND forces a line break at that point. So, if LaTeX puts 3 of 4
% authors names on the first line, and the last on the second line, try using
% \AND instead of \And before the third author name.

\author{
    Rylan Schaeffer \\
  Institute for Applied Computational Science\\
  Harvard University\\
  Cambridge, MA 02138 \\
   \texttt{rylanschaeffer@g.harvard.edu}
   \AND
   Leenoy Meshulam \\
  Brain and Cognitive Science\\
  Massachusetts Institute of Technology\\
  Cambridge, MA 02139 \\
   \texttt{leenoy@mit.edu}
  \And 
  Ila Fiete\\
  Brain and Cognitive Science\\
  Massachusetts Institute of Technology\\
  Cambridge, MA 02139 \\
  \texttt{fiete@mit.edu } \\
}

\begin{document}

\maketitle

\begin{abstract}
TODO
\end{abstract}

\section{Introduction}

\section{Normative Models}

Definitions. On trial $t$, let $b_t$ denote the block, $c_t$ denote the cue and $o_t$ the stimulus. Following common convention, let $T$ denote the total number of trials, $1:T$ denote all trials from the first to the last, and $<t$ denote all trials before the $t$-th trial.

The random variables are realized as follows:

\begin{align*}
    b_t &\sim p(b|b_{t-1})\\
    c_t &\sim p(c|b_t)\\
    o_t &\sim p(o|c_t)
\end{align*}{}

The generative model is 

\begin{figure}
    \centering
    \tikz{
        \node[latent] (b_t) {$b_t$} ;
        \node[latent, below=of b_t] (c_t)
        {$c_t$}
        \node[obs, below=of c_t] (o_t) {$o_t$}
        % \edge {b_t} {b_t}
        \edge {b_t} {c_t}
        \edge {c_t} {o_t}
    }
    \caption{Generative Model of IBL Task. On trial number $t$, $b_t$ is the block bias, $c_t$ is the cue, and $o_t$ is the observation.}
    \label{fig:my_label}
\end{figure}{}

\begin{figure}
  \centering
  \tikz{ %
    \node[latent] (alpha) {$\alpha$} ; %
    \node[latent, right=of alpha] (theta) {$\theta$} ; %
    \node[latent, right=of theta] (z) {z} ; %
    \node[latent, above=of z] (beta) {$\beta$} ; %
    \node[obs, right=of z] (w) {w} ; %
    \plate[inner sep=0.25cm, xshift=-0.12cm, yshift=0.12cm] {plate1} {(z) (w)} {N}; %
    \plate[inner sep=0.25cm, xshift=-0.12cm, yshift=0.12cm] {plate2} {(theta) (plate1)} {M}; %
    \edge {alpha} {theta} ; %
    \edge {theta} {z} ; %
    \edge {z,beta} {w} ; %
  }
\end{figure}



\section{Acknowledgments}

\section{References}
\printbibliography[heading=none]


\end{document}
